\documentclass{article}

\usepackage{graphicx} 
\usepackage{amsmath} 
\usepackage[T2A]{fontenc}
\usepackage[utf8]{inputenc}
\usepackage[russian,english]{babel}
\usepackage[left=2cm,right=2cm,top=2cm,bottom=2cm,footskip=1cm]{geometry}
\usepackage{xcolor}
\usepackage{hyperref}
\usepackage{amssymb}
\hypersetup{
    colorlinks=true,
    linkcolor=blue,
    filecolor=magenta,      
    urlcolor=cyan
}
\setlength\parindent{0pt} 

\renewcommand{\labelenumi}{\alph{enumi}.} 

\title{Семинар 1. \\ Машина Тьюринга и асимптотическая оценка рекуррент}

\author{Составил Р. Делла Пиетра} 
\date{8.2.20}

\begin{document}

\maketitle
\section{Машина Тьюринга}
\subsection{Устройство}
Машина Тьюринга состоит из бесконечной ленты, считывающей головки и правила переходов. В каждой ячейке ленты лежит некоторый символ из заданного конечного алфавита. Машина в каждый момент времени находится в одном состоянии из некоторого конечного количества возможных, считывающая головка над некоторой ячейкой. Такт работы состоит из считывания ячейки, просмотра таблицы переходов и опциональные перезапись ячейки и сдвиг считывающей головки вправо или влево на одну ячейку. \\
\[
\text{Формально: }
\left\{ \begin{array}{cl}
(A, Q, Q_f, q_0, \delta, \Lambda) & A\textit{~---~алфавит }\\
\Lambda \in A  & \Lambda \textit{~---~пустой символ, изначально в каждой ячейке ленты} \\
q_0 \in Q  & Q\textit{~---~множество состояний}\\
Q_f \subset Q  & q_0\textit{~---~начальное состояние} \\
|A| < \infty  & Q_f\textit{~---~подмножество конечных состояний} \\
|Q| < \infty \\
\delta: A \times Q \rightarrow A \times \{-1, 0 ,1\} \times Q
\end{array} \right.
\] 
\subsection{Тезис Чёрча-Тьюринга}
Главная идея машины Тьюринга в том, чтобы максимально просто с точки зрения устройства вычислительного аппарата применять математические рассуждения к теории алгоритмов. Тезис Чёрча-Тьюринга состоит в том, что любая фактически вычислимая функция, то есть для которой существует физическое устройство, её вычисляющее, моделируется на МТ. Это эмпирический факт, он никак не доказывается, но используя его можно проводить рассуждения такого вида: существуют двухчашечные весы, сравнивающие два объекта по весу, что по сути есть сравнение двух чисел, можно утверждать, что существует МТ $M$, которая принимает на входной ленте два числа и выдаёт ответ в виде $1$, если первое больше, $-1$ если второе больше и $0$, если они одинаковые. 
\subsection{Устройство}
Машина Тьюринга состоит из бесконечной ленты, считывающей головки и правила переходов. В каждой ячейке ленты лежит некоторый символ из заданного конечного алфавита. Машина в каждый момент времени находится в одном состоянии из некоторого конечного количества возможных, считывающая головка над некоторой ячейкой. Такт работы состоит из считывания ячейки, просмотра таблицы переходов и опциональные перезапись ячейки и сдвиг считывающей головки вправо или влево на одну ячейку. \\
\newpage
\subsection{Пример МТ}
Машина Тьюринга, определяющая чётность числа, записанного в унарной системе. 
\medskip \\
\begin{tabular}{ | l | l | l | l | l |}
	\hline
	 $\delta$ & $q_0$ & $q_1$ & $q_{even}$ & $q_{odd}$  \\ \hline
	 $\Lambda$ & $\langle \Lambda, 0, q_{even} \rangle$ & $\langle \Lambda, 0, q_{odd} \rangle $& & \\ \hline
	 a & $\langle a, 1, q_1 \rangle$ &  $\langle a, 1, q_0 \rangle$ & & \\ 
    \hline 
\end{tabular} 
\medskip \\
Легко видеть, если вход пустой (число $0$), то оно чётное, иначе мы шагаем и попеременно меняем $q_0$ и $q_1$. Если дошли до конца слова, и мы в состоянии $q_1$, значит было сделано нечётное количество шагов, и число нечётное. 
\subsection{Релевантные определения}
\textbf{\textit{Вычислимая функция}}: функция, для которой существует машина Тьюринга, её вычисляющая. Оказывается, не все функции вычислимы. \\
\textbf{\textit{Перечислимое множество}}: множество, для которого существует алгоритм, перечисляющий все элементы, возможно, с повторами. \\
\textbf{\textit{Разрешимое}}: множество, для которого существует вычислимая характеристическая функция: 
\[
\chi_{W}(x) = \left\{
\begin{array}{cc}
1 & x \in W \\
0 & x \not \in W 
\end{array} \right. 
\]
\subsection{Пример невычислимой функции}
\textbf{Проблема останова}: учитывая, что все объекты, которыми задаётся алгоритм на МТ конечны, пронумеруем все возможные алгоритмы $A_0, A_1, \dotsc$ и входы $I_0, I_1, \dotsc$. Функция $F_s(i, j) = \left\{
\begin{array}{cc}
1 & A_i \textit{ останавливается на } I_j \\
0 & \textit{иначе} 
\end{array} \right. $ невычислима. Доказательство можно посмотреть в \href{http://vyalyy.narod.ru/da3-100722.pdf}{учебнике по тфс}.
\newpage
\section{Асимптотические оценки рекуррент}
\subsection{Обозначения}
\[
\begin{array}{ccc}
f(x) = O(g(x)) &\iff& \exists c>0 \;  \exists x_0:\forall x > x_0  \; |f(x)| \leqslant c|g(x)| \\
f(x) = o(g(x)) &\iff& \forall c>0 \;  \exists x_0:\forall x > x_0  \; |f(x)| < c|g(x)| \\
f(x) = \Omega(g(x)) &\iff& \exists c>0 \;  \exists x_0:\forall x > x_0  \; c|g(x)| \leqslant |f(x)| \\
f(x) = \omega(g(x)) &\iff& \forall c>0 \;  \exists x_0:\forall x > x_0  \; c|g(x)| \leqslant |f(x)| \\
f(x) = \Theta(g(x)) &\iff& \exists c_1, c_2>0 \;  \exists x_0:\forall x > x_0  \; c_1|g(x)| \leqslant |f(x)| \leqslant c_2|g(x)| \\
\end{array}
\]
\subsection{Примеры}
\subsubsection{$T(n) = T(\sqrt n) + \log^3 n$}
$G(k) = T(2^k) \implies G(k) = G(\frac k 2) + k^3\log^3 2 = G(\frac k 4) + \frac{k^3}{8}\log^3 2 + k^3\log^3 2 = k^3(1 + \frac {1} {2^3} + \frac {1}{2^6} + \dotsc )\log^3 2$ \\
$G(k) \in \left(k^3\log^3 2, k^3 \sum\limits_{l = 0}^{\infty} \frac{1}{2^{3l}}\log^3 2\right) =  (Ak^3, Bk^3) \implies G(k) = \Theta(k^3) \implies \boxed{T(n) = \Theta (\log^3 n)}$

\subsubsection{$T(n) = \sqrt nT(\sqrt n) + n$}
$T(n) = n + \sqrt n (\sqrt n + n^{1/4}T(n^{1/4})) = 2n + n^{3/4}T(n^{1/4}) = \dotsc = 3n + n^{7/8}T(n^{1/8}) = \dotsc = kn + n^{1-2^{-k}}T(n^{2^{-k}})$.\\
Повторяться это будет, пока $n^{2^{-k}}$ не станет асимптотически стремиться к константе, \\
то есть $2^{-k} = \frac{1}{n}$ (потому что $n^{1/n} \rightarrow 1$). 
В итоге $\boxed{T(n) = \Theta(n\log n)}$

\subsubsection{$T(n) = T(\sqrt n) + n$}
$G(k) = T(2^k) \implies G(k) = G(\frac{k}{2}) + 2^k$ \\
$F(m) = G(2^m) \implies F(m) = F(m-1) + 2^{2^m} = \sum\limits_{l=0}^{m} 2^{2^l} > 2^{2^m}$ \\
Докажем, что $F(m) < 2\cdot2^{2^m}: \quad \sum\limits_{l=0}^{m-1} 2^{2^l} < \sum\limits_{p=0}^{2^{m} - 1} 2^{p} = 2^{2^m} - 1 \implies F(m) = \Theta(2^{2^m}) \implies \boxed{T(n) = \Theta(n)}$

\section{Полезные ссылки}
\href{https://vk.com/algo675?w=wall-139602945_5}{Семинар П. Останина (МТ + ссылки)}\\
\href{https://vk.com/algo675?w=wall-139602945_167}{Семинар П. Останина (Оценки, рекурренты)}\\
\href{https://vk.com/doc14914624_459299493?hash=21e416fbd55f65e820&dl=3ca48d6d91411a75c2}{Семинар А. Кулькова}\\



\end{document}

