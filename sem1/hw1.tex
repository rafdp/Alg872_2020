\documentclass[11pt,a5paper,fleqn]{article}
\usepackage{graphicx} 
\usepackage{amsmath} 
\usepackage[T2A]{fontenc}
\usepackage[utf8]{inputenc}
\usepackage[russian,english]{babel}
\usepackage[left=1cm,right=2cm,top=2cm,bottom=2cm,footskip=1cm]{geometry}
\usepackage{xcolor}
\usepackage{hyperref}
\usepackage{amssymb}
\hypersetup{
    colorlinks=true,
    linkcolor=blue,
    filecolor=magenta,      
    urlcolor=cyan
}
\setlength\parindent{0pt} 

\begin{document}


\begin{center}
{ \Large 1. Машина Тьюринга \\и асимптотическая оценка рекуррент}

\end{center}
{\bf 1.} Построить машину Тьюринга с одной или двумя лентами, распознающую палиндромы на алфавите $A = \{a, b, \Lambda\}$. \\
\smallskip
{\bf 2.} Докажите, что следующие определения перечислимого множества $X \subset \mathbb{N}$ эквивалентны:
\begin{itemize}
\item Существует алгоритм, печатающий все элементы множества (в любом порядке и со сколь угодно большими паузами между элементами).
\item Множество является областью определения некоторой вычислимой функции.
\item Множество является областью значений некоторой вычислимой функции.
\end{itemize}
\smallskip
{\bf 3.} Дан массив из $n$ элементов, на которых определено отношение равенства (например, речь может идти о массиве картинок или музыкальных записей). Постройте  алгоритм, который в <<потоковом режиме обработки данных>> \footnote{Поточный алгоритм (англ. streaming algorithm или on-line algorithm)~---~алгоритм для обработки последовательности данных в один или малое число проходов. В этой задаче предусматривается ровно два прохода.} 
определяет, есть ли в массиве элемент, повторяющийся больше $\frac{n}{2}$ раз. 
Считается, что в вашем распоряжении есть память объемом $O(\log n)$ битов. 

\newpage
{\bf 4.} На вход подается описания $n$ событий в формате $(s,f)$~---~время начала и время окончания. Требуется составить расписание для человека, который хочет принять участие в максимальном количестве событий. Например, события~---~это доклады на конфереции или киносеансы на фестивале, которые проходят в разных аудиториях. Предположим, что участвовать можно только с начала события и до конца. Рассмотрим три жадных алгоритма.
\begin{itemize}
        \item Выберем событие кратчайшей длительности, добавим его в расписание, исключим из рассмотрения события, \\
        пересекающиеся с выбранным. \\
        Продолжим делать то же самое далее.

        \item Выберем событие, наступающее раньше всех, добавим его в расписание, исключим из рассмотрения события, пересекающиеся с выбранным. Продолжим делать то же самое далее.

        \item Выберем событие, завершающееся раньше всех, добавим его в расписание, исключим из рассмотрения события, пересекающиеся с выбранным. Продолжим делать то же самое далее.

\end{itemize} 
      
      Какой алгоритм вы выберете? В качестве обоснования для каждой процедуры проверьте, что она является оптимальной (т.~е. гарантирует участие в максимальном числе событий) или постройте конкретный контрпример. 
 

\smallskip


{\bf 5 (Доп).} Найдите явное аналитическое выражение для производящей функции чисел $BR_{4n+2}$ правильных скобочных последовательностей длины $4n+2$ (ответ в виде суммы ряда не принимается). 

\smallskip

{\bf 6.} Оцените трудоемкость рекурсивного алгоритма, разбивающего исходную задачу размера $n$ на три задачи размером $\lceil\frac{n}{\sqrt{3}}\rceil-5$, используя для этого $10\frac{n^3}{\log n}$ операций.

\smallskip

{\bf 7 (Доп).}  Оцените как можно точнее глубину рекурсии для рекурренты
$T(n) = T(n-\lfloor \sqrt{n} \rfloor) + T(\lfloor \sqrt{n} \rfloor) + \Theta(n)$.


\end{document}



