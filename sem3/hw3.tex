\documentclass[11pt,a5paper,fleqn]{article}
\usepackage{graphicx} 
\usepackage{amsmath} 
\usepackage[T2A]{fontenc}
\usepackage[utf8]{inputenc}
\usepackage[russian,english]{babel}
\usepackage[left=1cm,right=2cm,top=2cm,bottom=2cm,footskip=1cm]{geometry}
\usepackage{xcolor}
\usepackage{hyperref}
\usepackage{amssymb}
\hypersetup{
    colorlinks=true,
    linkcolor=blue,
    filecolor=magenta,      
    urlcolor=cyan
}
\setlength\parindent{0pt} 

\begin{document}


\begin{center}
{ \Large 2. $\mathcal{P}$ vs $\mathcal{NP}$}

\end{center}
{\bf 1.} Язык 2-COLOR состоит из кодировок всех графов, заданных матрицами смежности, вершины которых можно корректно окрасить в два цвета (никакие две смежные вершины не имеют один цвет). Верно ли, что язык 2-COLOR лежит в $\mathcal{P}$? В $\mathcal{NP}$? В $co-\mathcal{NP}$?\\
\smallskip
{\bf 2.}  Язык $HP$ состоит из всех графов, имеющих гамильтонов путь (несамопересекающийся путь, проходящий через все вершины графа). Язык $HC$ состоит из всех графов, имеющих гамильтонов цикл (цикл, проходящий через все вершины, в котором все вершины, кроме первой и последней, попарно различны).
Постройте явные полиномиальные сводимости $HC$ к $HP$ и наоборот.\\
\smallskip
{\bf 3.}  Покажите, что язык всех тавтологичных 3-КНФ является полным в классе $co-\mathcal{NP}$. Верно ли это для языка всех тавтологичных 2-КНФ? \\
\smallskip
{\bf 4.}  Докажите следующие свойства полиномиальной сводимости:

($i$) Рефлексивность: $A\leq_p A$; транзитивность: $A\leq_p B, B\leq_p C \implies A\leq_p C$;

($ii$) Если $B\in\mathcal{P}$ и $A\leq_p B$, то $A\in\mathcal{P}$;

($iii$) Если $B\in\mathcal{NP}$ и $A\leq_p B$, то $A\in\mathcal{NP}$.\\
\smallskip
{\bf 5.}  Докажите, что классы $\mathcal{P}$ и $\mathcal{NP}$ замкнуты относительно \\
операции $*$~---~звезды Клини (была в ТРЯПе). Приведите также и сертификат принадлежности слова языку $L^*$, где $L\in\mathcal{NP}$. \\
\smallskip
{\bf 6 (Доп)}  Верно ли, что класс $co-\mathcal{NP}$ замкнут относительно операции чётной итерации \\
$L^{even-*} = \{\varepsilon\} \cup L^2 \cup L^4 \cup\dotsc$? \\
\smallskip
{\bf 7 (Доп)} Замкнут ли класс $\mathcal{P}$ относительно взятия подслова? \\
\end{document}



