\documentclass[11pt,a5paper,fleqn]{article}
\usepackage{graphicx} 
\usepackage{amsmath} 
\usepackage[T2A]{fontenc}
\usepackage[utf8]{inputenc}
\usepackage[russian,english]{babel}
\usepackage[left=1cm,right=2cm,top=2cm,bottom=2cm,footskip=1cm]{geometry}
\usepackage{xcolor}
\usepackage{hyperref}
\usepackage{amssymb}
\hypersetup{
    colorlinks=true,
    linkcolor=blue,
    filecolor=magenta,      
    urlcolor=cyan
}
\setlength\parindent{0pt} 

\begin{document}


\begin{center}
{ \Large 2. Перечислимость, разрешимость, $m$-сводимость}

\end{center}
{\bf 1.} Функция $u(M)$ равна наибольшему числу тактов работы на входных словах длины $10$, если МТ $M$ останавливается на каждом таком слове, и не определена в противном случае. Вычислима ли $u(M)$?\\
\smallskip
{\bf 2.}  Разрешим ли язык $L$, состоящий из всех описаний МТ, у которых есть недостижимое состояние (не достигается ни при каком входе)?\\
\smallskip
{\bf 3.}  Перечислим ли язык $L_\emptyset$ состоящий из всех описаний МТ, которые не останавливаются ни на каком входе? \\
\smallskip
{\bf 4.}  Показать, что любой перечислимый язык сводится к $L_{stop}$.\\
\smallskip
{\bf 5.}  Верно ли, что все непустые коперечислимые языки $m$-сводятся друг к другу?\\
\smallskip
{\bf 6}  Функция Трудолюбия Радо (busy beaver function) определяется, как максимальное количество единиц, которые может напечатать МТ с $n$ состояниями перед остановкой. 
\begin{itemize}
\item Всюду ли эта функция определена?
\item (Доп) Вычислима ли эта функция?
\end{itemize}
\smallskip
{\bf 7 (Доп).}  Постройте биекции: 
\begin{itemize}
\item $(0, 1) \rightarrow (0, +\infty)$
\item $[0, 1] \rightarrow [0, 1)$
\item $[0, 1] \rightarrow [0, 1]^2$
\item $2^\mathbb{N} \rightarrow [0, 1]$
\end{itemize}
\end{document}



